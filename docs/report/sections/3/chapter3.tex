\title{Controllo non lineare con il metodo di Ljapunov}
\maketitle
\label{sec:nonlinear-control}


\paragraph{Introduzione.}

In questo capitolo introduco i concetti di
controllabilità asintotica e di funzione di controllo di
Ljapunov\footnotemark.
Dimostro poi il risultato principale
che si ottiene dalle funzioni di Ljapunov, ovvero,
dimostro che trovare una funzione di Ljapunov per un sistema
ne garantisce la controllabilità.

\footnotetext{%
Aleksandr Michajlovič Ljapunov (1857 – 1918) è stato un matematico e fisico russo, noto soprattutto per i suoi risultati sulla stabilità dei sistemi dinamici.
Il suo nome viene traslitterato anche come Lyapunov, Liapunov o Ljapunow.
}

\input{sections/3/stabilità-asintotica}
\section{Funzioni di controllo di Ljapunov}
Per i sistemi non lineari esiste un criterio che permette di determinarne
la controllabilità asintotica, chiamato \emph{metodo di Ljapunov}.
L'idea è di trovare una \emph{energia generalizzata} $V$
che sia sempre decrescente nel tempo,
così che il sistema tenda a minimizzarla.
La difficoltà nell'applicare questo metodo sta nel trovare una funzione di Lyapunov;
compito non facile, in generale, e per cui spesso è richiesta
una qualche forma di \emph{ispirazione Divina}~\cite{strogatz}.


Per usare i risultati della teoria di Ljapunov è necessario lavorare
con \emph{sistemi topologici}, una classe più ristretta di problemi di controllo.
Per quanto riguarda questo testo è sufficiente pensare a un sistema topologico
come un problema di controllo in cui la mappa di transizione è una funzione
continua.
% Anche qui citare sontag

\begin{definition}
    Un \textbf{sistema topologico} è un problema di controllo in
    cui lo spazio delle fasi $\Sigma$ è dotato di metrica $d$ e,
    dato $t \in \mathcal T$ e $\b x \in \Sigma$,
    vale la seguente proprietà:

    \emph{Se $\omega$ è un controllo ammesso per $\b x$ e la successione
    ${\b x_n}$ tende a $\b x$ allora deve esistere un intero
    $N$ tale che, preso $n \geq N$ allora $\omega$ è ammesso per $x_n$
    e il limite~\eqref{eq:def-sistema-topologico} converge uniformemente a $0$ per $s \in [0, t]$.}
    \begin{equation}
        \lim_{N \to +\infty} d[\phi^s(\b x_n, \omega), \phi^s(\b x, \omega)] = 0.
        \label{eq:def-sistema-topologico}
    \end{equation}
\end{definition}


Enuncio la definizione di \emph{funzione di controllo di Ljapunov}.
\begin{definition}
    \label{def:clf-local}
    Dato un problema di controllo in cui $\b x^*$ è un punto fisso e
    $\mathcal O$ è un intorno di $\b x^*$.
    Una funzione continua
    \begin{equation*}
        V : \Sigma \to \R,
    \end{equation*}
    è detta \textbf{funzione locale di controllo di Ljapunov} se
    e solo se valgono le seguenti proprietà:
    \begin{itemize}
        \item 1) $V$ è propria in $\b x^*$, ovvero l'insieme
        \begin{equation*}
            \left\{\b x \in \Sigma\ |\ V(\b x) \leq \epsilon \right\}
        \end{equation*}
        è un sottoinsieme compatto di $\mathcal O$ per ogni $\epsilon$ piccolo a piacere.
        \item 2) $V$ è definita positiva in $\mathcal O$, ovvero $V(\b x^*) = 0$ e
        $V(\b x) > 0$ per ogni $\b x \neq \b x^* \in \mathcal O$.
        \item 3) Per ogni $\b x \neq \b x^* \in \mathcal O$
        esiste un tempo $0 < t \in \mathcal T$ e un controllo $\omega \in \mathcal U^{[0,t[ \subseteq \mathcal T}$
        tali che, detta $\xi(s) = \phi^s(\b x, \omega)$, vale
        \begin{align*}
            V(\xi(s)) &\leq V(\b x) \text{ per ogni } s \in [0,t[ \subseteq \mathcal T, \\
            V(\xi(t)) &< V(\b x).
        \end{align*}
    \end{itemize}
\end{definition}

\begin{definition}
    Se nella Definizione~\ref{def:clf-local} l'intorno $\mathcal O$ coincide
    con tutto lo spazio delle fasi $\Sigma$ allora $V$ è una
    \textbf{funzione di controllo di Ljapunov}.
    \label{def:clf}
\end{definition}

Le Definizioni~\ref{def:clf-local} e~\ref{def:clf} valgono anche per sistemi
senza controlli; in questo caso $V$ è chiamata solamente \emph{funzione (locale) di Ljapunov}.
Il termine (\emph{locale}) tra parentesi indica che l'affermazione è da intendersi valida
sia che il termine \emph{locale} sia presente o meno.

\begin{definition}
    \label{def:ben-raggiungibile}
    Sia $V$ una funzione (locale) di controllo di Ljapunov.
    Dati $\b x, \b z \in \Sigma$, si dice che $\b z$ è \textbf{ben raggiungibile
    da} $\b x$ se esiste un tempo $t > 0$ e un controllo $\omega \in \mathcal U^{[0,t[ \subseteq \mathcal T}$ ammesso per $\b x$
    tale che per $\xi(s) = \phi^s(\b x, \omega)$ con $\xi(t) = \b z$, valgono
    \begin{align*}
        V(\xi(s)) &\leq V(\b x) \text{ per ogni } s \in ]0,t[ \subseteq \mathcal T,~\\
        V(\xi(t)) &< V(\b x).
    \end{align*}
    \vspace{-1cm}
\end{definition}

Il risultato principale che si ottiene dalle funzioni di Ljapunov è racchiuso nel seguente teorema.
\begin{thm}[Teorema di controllabilità di Ljapunov]
    Dato un sistema topologico, se esiste una funzione (locale) di controllo di Ljapunov,
    allora il problema è (localmente) asintoticamente controllabile.
    \label{thm:ljapunov}
\end{thm}

\emph{Dimostrazione}.
La dimostrazione si basa sulla costruzione di una strategia di
controllo che porti il sistema verso $\b x^*$ in un tempo infinito.
Per chiarezza, divido la dimostrazione in più passaggi; dimostro prima la versione locale
del teorema e poi quella globale.
\begin{steps}
    \item Grazie alla prorpietà (1) della Definizione \ref{def:clf-local}
    posso scegliere un numero $\alpha_0$ tale che l'insieme
    \begin{equation}
        C = \left\{ x\ |\ V(x) \leq \alpha_0 \right\} \subset \mathcal O
        \label{eq:alpha_0}
    \end{equation}
    sia \emph{compatto}.

    Dimostro la seguente affermazione:

    \begin{aff}
        Per ogni intorno aperto $\mathcal W$ di $\b x^*$ esiste
        un $\beta > 0$ tale che
        \begin{equation*}
            \{\b x\ |\ V(\b x) \leq \beta\} \subset \mathcal W.
        \end{equation*}
    \end{aff}

    \emph{Dimostrazione}.
    Procedo per assurdo.
    Se così non fosse, esisterebbe una sequenza di elementi di $\Sigma$
    \begin{equation*}
    \{\b x_n\}
    \end{equation*}
    tali che per ogni $n$
    valgano
    \begin{align*}
        &\b x_n \notin \mathcal W, \\
        &V(\b x_n) \to 0 \text{ per } n \to +\infty.
    \end{align*}
    Senza perdere di generalità, assumo che tutti gli $\b x_n$
    siano contenuti in $K$, dato da
    \begin{equation*}
        K = \bar{\mathcal W} \bigcap C,
    \end{equation*}
    dove $\bar{\mathcal W}$ indica il complementare di $\mathcal W$.
    La costruzione per la dimostrazione è mostrata in \autoref{fig:ljapunov-dim-aff1}.

    \hfill
    \begin{minipage}{.8\textwidth}
        \begin{figure}[H]
            \centering
            \includegraphics[width=\textwidth,clip,trim=2.2cm 1cm 2.2cm 0]{assets/ljapunov-dim-aff1}
            \caption[Costruzione 1 per teorema di Ljapunov]{Costruzione per dimostrare
            l'affermazione 1.}
            \label{fig:ljapunov-dim-aff1}
        \end{figure}
    \end{minipage}

    Per il teorema di Bolzano-Weierstrass\footnotemark, posso estrarre una
    sottosuccessione di $\{\b x_n\}$ convergente in $K$
    \begin{equation*}
    \{\b x_ {n_k}\} \to \b x.
    \end{equation*}
    Quindi, per la continuità di $V$ vale
    \begin{equation*}
        V(\b x) = 0
    \end{equation*}
    e dalla proprietà (2) della Definizione~\ref{def:clf-local}
    segue che
    \begin{equation*}
        \b x = \b x^*.
    \end{equation*}
    Ma questo è un assurdo, visto che che $\b x \in K$
    e $\b x^* \in \mathcal W \subset \bar K$, e prova l'affermazione.

    \hfill\openbox\paragraph{}

    \footnotetext{%
    Karl Theodor Wilhelm Weierstrass (1815 – 1897) è stato un matematico tedesco, spesso chiamato padre dell’analisi moderna. Si occupò di definire rigorosamente i fondamenti dell’analisi, dando per primo l’esempio di una funzione continua ovunque ma non derivabile.
    
    Bernard Placidus Johann Nepomuk Bolzano (1781 – 1848) è stato un matematico, filosofo, teologo, presbitero e logico boemo che diede significativi contributi sia alla matematica che alla teoria della conoscenza.
    }

    Una conseguenza utile dell'affermazione (1) è che se
    $\xi(t) = \phi^t(\b x, \omega)$ con $t \in [0, +\infty[$ tale che
    \begin{equation*}
        V[\xi(t)] \to 0 \text{ per } t \to +\infty
    \end{equation*}
    allora deve valere
    \begin{equation}
        \xi(t) \to \b x^*.
        \label{eq:xi-to-xstar}
    \end{equation}


    \item Osservo che la proprietà (3) della Definizione~\ref{def:clf-local}
    garantisce che per ogni stato $\b y \in \mathcal O$ e $\b y \neq \b x^*$ esiste
    almeno uno stato ben raggiungibile da $\b y$.
    Inoltre vale la proprietà transitiva, ovvero, se $\b y$ è raggiungibile da $\b x$
    e $\b z$ è raggiungibile da $\b y$, allora $\b z$ è raggiungibile da $\b x$.
    Definisco
    \begin{equation*}
        B(\b x) = \inf\left\{ V(\b z)\ |\ \b z \text{ è ben raggiungibile da } \b x \right\}
    \end{equation*}
    e osservo che, per quanto appena detto, vale che $B(x) < V(x)$ per $\b x \neq \b x^*$.
    Dimostro la seguente affermazione.

    \begin{aff}
        \begin{equation*}
            V(\b x) < \alpha_0 \implies B(\b x) = 0,
        \end{equation*}
        dove $\alpha_0$ definisce un insieme compatto, secondo la~\eqref{eq:alpha_0}.
    \end{aff}

    \emph{Dimostrazione}.
    Procedo per assurdo.
    Suppongo che esista un $\b x \neq \b x_0$ per cui $V(\b x) < \alpha_0$
    e $B(\b x) = \alpha > 0$.
    Per come ho scelto $\alpha_0$ deve valere $\b x \in \mathcal O$.
    Considero la successione\footnote{L'esistenza
    è garantita dalla proprietà transitiva della ben raggiungibilità e dal fatto che
    ogni $\b z_n$ ha almeno un elemento che è ben raggiungibile.}
    \begin{equation}
        \{V(\b z_n)\}
        \label{eq:successione-zn}
    \end{equation}
    decrescente dove gli $\b z_n$ sono tutti elementi ben raggiungibili da $\b x$.
    La~\eqref{eq:successione-zn} è limitata da $V(\b z) = \alpha$.
    Tutti gli $\b z_n$ fanno parte del compatto
    \begin{equation*}
        C = \left\{\b z \ |\ V(\b z) \leq V(\b x) \right\}
    \end{equation*}
    e dato che la~\eqref{eq:successione-zn} è monotona e limitata,
    allora è anche convergente e posso fissarne il limite senza perdere
    di generalità:
    \begin{equation}
        \lim_{\b z_n \to \b z} V(\b z_n) = V(\b z) = \alpha.
        \label{eq:vzequalsalpha}
    \end{equation}
    Osservo che
    \begin{equation*}
        \alpha \neq 0 \implies \b z \neq \b x^*
    \end{equation*}
    e che
    \begin{equation*}
        \alpha < V(\b x) < \alpha_0 \implies \b z \in \mathcal O.
    \end{equation*}
    Quindi, deve esistere un $\b y$ che sia ben raggiungibile da $\b z$.
    È importante osservare che, anche se ogni elemento della successione~\eqref{eq:successione-zn}
    è ben raggiungibile da $\b x$, non ho nulla che mi garantisca che $\b z$ lo sia.
    Fisso $\epsilon > 0$ tale che
    \begin{equation}
        V(\b z) < V(\b x) - \epsilon,\ \text{e} \ V(\b y) < V(\b z) - \epsilon
        \label{eq:vy-less-vz}
    \end{equation}
    e prendo $\nu \in \mathcal U^{[0, t[ \subseteq \mathcal T}$ controllo ammesso
    per $\b z$ tale che, detto $\zeta(s) = \phi^s(\b z, \nu)$, valgano
    \begin{align*}
        &\zeta(t) = \b y, \\
        &V[\zeta(s)] \leq V(\b z), \ \forall s \in [0, t[ \subseteq \mathcal T.
    \end{align*}
    La costruzione per la dimostrazione è mostrata in \autoref{fig:ljapunov-dim-aff2}.

    \hfill
    \begin{minipage}{.8\textwidth}
        \begin{figure}[H]
            \centering
            \includegraphics[width=.7\textwidth]{assets/ljapunov-dim-aff2}
            \caption[Costruzione 2 per teorema di Ljapunov]{Costruzione per dimostrare
            l'affermazione 2.}
            \label{fig:ljapunov-dim-aff2}
        \end{figure}
    \end{minipage}

    $V$ è uniformemente continua\footnote{Per ipotesi
        $V$ è continua sul compatto $C$ quindi per il teorema di Heine-Cantor è
        uniformemente continua su $C$.} sul compatto $C$ quindi, fissato $\epsilon >0$ esiste un $\delta > 0$
    tale che, presi $\b a, \b b \in C$
    \begin{equation}
        d(\b a, \b b) < \delta \implies |V(\b a) - V(\b b)| < \epsilon.
        \label{eq:v-uniformemente-continua}
    \end{equation}

    Considero la successione
    \begin{equation*}
        \{V(\b y_n)\}, \text{ con } \b y_n = \zeta_n(t)
    \end{equation*}
    dove $\zeta_n(s) = \phi^s(\b z_n, \nu)$.

    Visto che sto lavorando su uno spazio topologico vale che, fissato $\epsilon > 0$, esiste $\delta > 0$ tale che,
    per valori $n \geq N$ vale
    \begin{equation}
        d_\infty(\zeta_n, \zeta) < \delta \implies d(\b y_n, \b y) < \epsilon,
        \label{eq:phi-continua}
    \end{equation}
    dove $d_\infty$ è la metrica per lo spazio dei controlli ammessi definita da
    \begin{equation*}
        d_{\infty}[\phi^t(\b x, \omega_1), \phi^t(\b x, \omega_2)] =
        \sup \left\{ d[\phi^s(\b x, \omega_1), \phi^s(\b x, \omega_2)],\ s \in \mathcal T  \right\}.
    \end{equation*}
    Ora fisso $\epsilon > 0$ e considero la catena di implicazioni data
    dalla~\eqref{eq:v-uniformemente-continua} e dalla~\eqref{eq:phi-continua}:
    \begin{equation}
        d_\infty(\zeta_n, \zeta) < \delta' \implies d(\b y_n, \b y) < \delta \implies |V(\b y_n) - V(\b y)| < \epsilon
        \label{eq:implication-chain}
    \end{equation}
    da cui
    \begin{equation}
        \left| V[\zeta_N(s)] - V[\zeta(s)] \right| < \epsilon\ \forall s \in [0, t].
        \label{eq:lesser-forall-t}
    \end{equation}

    Dimostro che $\b y_N$ è ben raggiungibile da $\b x$.
    È sufficiente dimostrare che
    \begin{equation*}
        V[\zeta_N(s)] < V(\b x)\  \forall s \in [0, t].
    \end{equation*}
    Considero la~\eqref{eq:lesser-forall-t} e applico la definizione di modulo (studio entrambi i casi).
    \begin{itemize}
        \item Caso $V[\zeta(s)] < V[\zeta_N(s)]$:
            \begin{align*}
                V[\zeta_N(s)] - V[\zeta(s)] &< \epsilon \\
                V[\zeta_N(s)] &< \epsilon + V[\zeta(s)] \\
                V[\zeta_N(s)] &< \epsilon + V(\b z) \\
                V[\zeta_N(s)] &< V(\b x) \numberthis\label{eq:modulo-caso-1}
            \end{align*}
            dove nell'ultimo passaggio ho applicato la~\eqref{eq:vy-less-vz}.
        \item Caso $V[\zeta(s)] > V[\zeta_N(s)]$:
            \begin{align*}
                V(\b x) > V(\b z) > V[\zeta(s)] > V[\zeta_N(s)]. \numberthis\label{eq:modulo-caso-2}
            \end{align*}
    \end{itemize}
    Chiamo $\omega$ la legge di controllo che porta $\b x$ a $\b z_N$.
    Per la proprietà di composizione della mappa di transizione, posso concatenare
    $\omega$ con $\nu$, in modo da ottenre una legge di controllo che porti $\b x$ a $\b y_N$.
    Questo, unito al fatto che la~\eqref{eq:modulo-caso-1} e la~\eqref{eq:modulo-caso-2}
    valgono per $s \in [0, t]$, è sufficiente a dimostrare che $\b y_N$ è ben raggiungibile da $\b x$.

    In questo modo cado in assurdo visto che la~\eqref{eq:implication-chain}
    unita alla~\eqref{eq:vy-less-vz} e alla~\eqref{eq:vzequalsalpha} implicano
    che $V(\b y_n) < B(\b x)$, violando la tesi.
    L'assurdo è dato dall'aver assunto che $\alpha = B(\b x) > 0$.

    \hfill \openbox \paragraph{}

    L'andamento di $B(\b x)$ è mostrato in \autoref{fig:ljapunov-aff2}.

    \hfill
    \begin{minipage}{.8\textwidth}
        \begin{figure}[H]
                \centering
                \includegraphics[width=\textwidth]{assets/ljapunov-aff2}
                \caption[Andamento di $B(\b x)$]{Andamento di $B(x)$ per uno
                spazio delle fasi monodimensionale. La funzione è a tratti e
                si annulla dentro al compatto più grande contenente $x^*$.}
                \label{fig:ljapunov-aff2}
        \end{figure}
    \end{minipage}

    \item Dimostro che vale la seguente affermazione

    \begin{aff}
        Se $V(\b x) < \alpha_0$ con $\b x \neq \b x^*$ allora esiste
        una sequenza di stati
        \begin{equation*}
        \{\b x_n\}
        \end{equation*}
        con $\b x_0 = \b x$, una sequenza di tempi
        \begin{equation*}
        t_n \in \mathcal T
        \end{equation*}
        e una sequenza di controlli
        \begin{equation*}
            \omega_n \in \mathcal U^{[0, t_n[ \in \mathcal T}
        \end{equation*}
        tali che valgano le seguenti proprietà
        \begin{itemize}
            \item $\omega_n$ è ammesso per $\b x_n$.
            \item $\phi^{t_n}(\b x_n, \omega_n) = \b x_{n+1}$
            \item Con $\xi_n(t) = \phi^t(\b x_n, \omega_n)$ vale $V[\xi_n(t)] \leq \frac 1 {2^n} V(\b x), \ \forall t \in [0, t_n]$
        \end{itemize}
    \end{aff}

    \emph{Dimostrazione}.
    Procedo per induzione.
    Voglio dimostrare che  per ogni $\b x \in \Sigma$ tale che
    \begin{equation*}
        0 < V(\b x) < \alpha_0
    \end{equation*}
    esiste un certo tempo $t > 1$ e un controllo $\omega$
    di lunghezza $t$ ammesso per $\b x$ tale che
    \begin{align*}
        V[\xi(s)] &\leq V(\b x) \text{ con } s \in [0, t], \\
        V[\xi(t)] &< \frac 1 2 V(\b x).
    \end{align*}
    Per continuità di $V$ in $\b x^*$, esiste un $\epsilon > 0$ per cui
    \begin{equation}
        d(\b z, \b x^*) < \epsilon \implies V(\b z) < \frac 1 2 V(\b x).
        \label{eq:dzxstar}
    \end{equation}
    Sia $\omega_0 \in \mathcal U^{[0, 1[}$ il controllo nullo,
    tale che
    \begin{equation*}
        \phi^s(\b x^*, \omega_0) = \b x^*.
    \end{equation*}
    Visto che sto lavorando in uno spazio topologico
    esiste un $\delta > 0$ tale che, fissato $\epsilon > 0$ e
    dato $\zeta(s) = \phi^s(\b y, \omega_0)$,
    \begin{equation*}
        d(\b y, \b x^*) < \delta \implies d(\zeta(s), \b x^*) < \epsilon
    \end{equation*}
    e $\omega_0$ è ammesso per $\b y$ per $s \in \mathcal T$.

    Per la~\eqref{eq:dzxstar} posso scegliere $\epsilon$ in modo che
    \begin{equation*}
        V[\zeta(s)] < \frac 1 2 V(\b x)
    \end{equation*}
    valga per $s \in [0, 1]$.
    Per l'affermazione (1) esiste $\delta_0 > 0$ tale che
    \begin{equation}
        V(\b y) < \delta_0 \implies d(\b y, \b x^*) < \delta.
        \label{eq:vylessdelta0}
    \end{equation}

    Per l'affermazione (2), vale $B(\b x) = 0$.
    Quindi esiste un controllo $\omega_1$ e un tempo $t_1$ tale
    che, dato $\xi_1 = \phi^{t_1}(\b x, \omega_1)$, vale
    \begin{align*}
        V[\xi_1(s)] &\leq V(\b x), \text{ con } s \in [0, t_1],\\
        V[\xi_1(t_1)] &< \delta_0
    \end{align*}
    e, per la~\eqref{eq:vylessdelta0}, vale anche
    \begin{equation*}
        d[\xi_1(t_1), \b x^*] < \delta.
    \end{equation*}
    La costruzione per la dimostrazione è mostrata in \autoref{fig:ljapunov-dim-aff3}.

    \hfill
    \begin{minipage}{.8\textwidth}
        \begin{figure}[H]
            \centering
            \includegraphics[width=.8\textwidth]{assets/ljapunov-dim-aff3}
            \caption[Costruzione 3 per teorema di Ljapunov]{Costruzione per dimostrare
            l'affermazione 3.}
            \label{fig:ljapunov-dim-aff3}
        \end{figure}
    \end{minipage}

    La sequenza che cerco è quindi data dalla concatenazione di
    $\omega_1$ e $\omega_0$ opportunamente traslati.

    \hfill\openbox\paragraph{}

    L'affermazione (3) mi permette di definire una legge di controllo
    $\omega$ su $[0, +\infty[ \subseteq \mathcal T$ ammessa per $\b x$
    data dalla concatenazione degli $\omega_n$.
    Detta quindi $\xi(t) = \phi^t(\b x, \omega)$ vale
    \begin{equation*}
        V[\xi(t)] \to 0 \text{ per } t \to +\infty
    \end{equation*}
    e per la~\eqref{eq:xi-to-xstar} vale
    \begin{equation}
        \xi(t) \to \b x^*.
        \label{eq:xi-to-xstar-2}
    \end{equation}

    \item Dimostro che il sistema è localmente asintoticamente
    controllabile.
    Prendo $\mathcal V$ un intorno di $\b x^*$, prendo $\alpha_1$ tale che
    \begin{equation}
     \{\b y\ |\ V(\b y) < \alpha_1 \} \subseteq \mathcal V
        \label{eq:def-alpha1}
    \end{equation}
    e definisco
    \begin{equation*}
        \alpha = \min\{\alpha_0, \alpha_1\}.
    \end{equation*}
    Nella definizione di localmente asintoticamente controllabile prendo
    \begin{equation*}
            \mathcal W = \{\b y\ |\ V(\b y) < \alpha\}.
    \end{equation*}

    L'affermazione (3) è valida per ogni $\b y \neq \b x^* \in \mathcal W$
    dato che $V(\b y) < \alpha_0 \leq \alpha$.
    Quindi, per la~\eqref{eq:xi-to-xstar-2} esiste un controllo $\omega$
    su $\mathcal T$ tale che
    \begin{align*}
        &\phi^0(\b y, \omega) = \b y, \\
        &\phi^t(\b y, \omega) \to \b x^* \text{ per } t \to +\infty
    \end{align*}
    e per la~\eqref{eq:def-alpha1} vale anche
    \begin{equation*}
        \phi^t(\b y, \omega) \in \mathcal V, \ t \in [0, +\infty[ \subseteq \mathcal T.
    \end{equation*}
    quindi il sistema è localmente asintoticamente controllabile.

    \hfill\qedsymbol\paragraph{}

    \item Dimostro che il sistema è asintoticamente controllabile.
    Assumo che $V$ sia una funzione di controllo di Ljapunov globale
    e prendo $\b y \in \Sigma$.
    Sia $\beta = V(\b y)$.
    Ora osservo che nella dimostrazione fatta fino ad ora posso prendere
    $\mathcal O = \Sigma$ e $\alpha_0 = \beta + 1$
    senza perdere di generalità, in quanto l'unica proprietà che $\mathcal O$
    e $\alpha_0$ devono rispettare è che l'insieme $C$
    definito nella~\eqref{eq:alpha_0}
    sia compatto e contenuto in $\mathcal O$.
    Quindi, se nel passaggio (4) prendo $\mathcal V = \Sigma$
    posso prendere $\alpha_1 = \beta + 1$ in modo che
    $\b y \in \mathcal{W}$ e che il sistema sia
    asintoticamente controllabile a $\b x^*$.

    \hfill\qedsymbol\paragraph{}

\end{steps}

Concludo questo capitolo dicendo che è possibile
dimostrare la seguente proposizione.
\begin{prop}
    \label{prop:condizione-sufficiente-ljapunov}
    Sia dato uno spazio topologico in cui è 
    definita una funzione $V : \Sigma \to \R$ continua.
    Sia $\mathcal O \subseteq \Sigma$
    un insieme aperto per cui la restrizione di
    $V$ su $\mathcal O$ sia differenziabile con 
    continuità e valgano le proprietà (1) e (2) 
    della Definizione~\ref{def:clf-local}.
    Allora, una condizione sufficiente perché $V$
    sia una funzione di controllo di Ljapunov locale è che per ogni $\b x \neq \b x_0 \in \mathcal O$
    esista un controllo $\omega$ tale che
    \begin{equation*}
        \frac{dV(\phi^t(\b x, \omega))}{dt} < 0.
    \end{equation*}
    Se questa proprietà vale per $\mathcal O = \Sigma$, $V$ è una funzione di controllo di Ljapunov globale.
\end{prop}
La dimostrazione si trova in~\cite{sontagMath}.
