\chapter{Conclusioni}
In questa tesi ho mostrato che lo studio della
controllabilità di un sistema semplice come il
pendolo su rotaia richiede l'utilizzo di risultati importanti della
Teoria del Controllo.
Ho anche mostrato l'applicabilità della Teoria nel mondo reale e
l'efficacia delle due strategie di controllo proposte,
grazie alla creazione di un esperimento in laboratorio.
Lo studio che ho svolto è solamente un punto di partenza
e può procedere in due possibili direzioni.
La prima, legata alla fisica, è lo studio più approfondito
della Teoria del Controllo
applicata a sistemi non lineari e caotici,
che presenta delle difficoltà
più elevate rispetto alla controparte per sistemi lineari.
La seconda, legata al mondo della didattica, è lo studio di come
la dimostrazione
pratica di concetti teorici possa essere uno strumento utile per migliorare il
coinvolgimento degli studenti verso una certa disciplina.
